
\textbf{Series}

We discuss the meaning of $ \sum_{n=0}^{\infty} a_n = S $

\textbf{Cauchy} 
Since it is a sequence we can use the Cauchy criterion to estimate a sum

\textbf{Condition} $  p =0  \implies \abs{a_m} < \epsilon $ which is equivalent to $ \lim\limits_{n \rightarrow \infty} a_n  $

\textbf{Remark} $ R_n = \sum_{i + n+1}^{} a_i $

Observe that the following converges:

\[ \sum_{}^{\infty} \frac{(-1)^n}{n} = 1 - \frac{1}{2} + \frac{1}{3} - \frac{1}{4} \ldots \]

\textbf{Absolute Convergence}
\[ \sum_{n = 0}^{\infty} a_n \] converges absolutely if and only if (def) $ \sum_{n=0}^{\infty} \abs{a_n} $ converges.

\textbf{Lemma} Absolute converges implies convergence

\textit{Proof} Check Cauchy. 

\textbf{The Comparison test}
If $ 0 \leq a_k \leq b_k $, then $ \sum b_k \qt{conv}  \implies \sum a_k \qt{conv}$

\textbf{Arithmetic}
Let us suppose we have a series on the following way: $  \sum^\infty  $ converges or converges absolutely if and only if $ \sum^\infty \bar{a_n} $ and $ \bar{\sum^\infty a_n} = \sum^\infty \bar{a_n} $.

\textbf{2.} Suppose we have two series that conv. or absolutely converges. $ \sum^\infty a_n \quad \sum^\infty b_n $ implies that $ \sum^\infty (a_n + b_n) $ converges or absolutely, and $ \sum (a_n + b_n) = \sum a_n + \sum b_n $.

\textbf{3.} Given a complex number $ \lambda \in \CC $, and $ \sum_{}^{\infty} a_n$ converges or converges absolutely, then $ \sum_{}^{\infty} \lambda a_n  = \lambda \sum_{}^{\infty} a_n$.

\textbf{4.}(Product)
We define the Cauchy product $ (a_n),(b_n) $ the Cauchy product is: $ c_n = \sum_{i=0}^{n} a_ib_{n-i} $

\textbf{Proposition} $ \sum_{}^{\infty} a_n, \sum_{}^{\infty} $ is absolutely convergent, then $ \sum c_n $ is also absol. convergent, and $ \sum c_n = \sum a_n \cdot \sum b_n $.

Using a proof with draw.

\textbf{Sequence of functions} $ \CC \supset E \rightarrow \CC $ $ \function{(f_n)}{f_0,f_1, \ldots}{\CC} $ 
Suppose that for all $ a \in E (f_n(a)) $ converges.

Study pointwise convergence to $f$. 
Study also uniform convergence.
