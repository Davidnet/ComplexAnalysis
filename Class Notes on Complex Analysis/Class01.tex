\section{Complex Numbers}
We want to solve a quadratic equation:
\[ ax^2 + bx + c = 0 \]
so we first calculate the discriminant:
\[ D =  b^2 - 4ac \]
it is called that, because it differentiate solution types. We then compute:
\[ r_{\pm} \frac{-b \pm \sqrt{D} }{2a}  \]
In $ R $ squares are positives,  but this poses a problem when $D < 0$. Meanwhile, let us observe that when $ D > 0 $ we get two solutions, and when we get $ D = 0$ there is only one solution. So the main question is what to do when $ D < 0$. Without loss of generality we can assume that $ a=1$
\begin{align*}
r_+ = -\frac{b}{2} + \sqrt{2} \\
r_- = -\frac{b}{2} - \sqrt{2}
\end{align*}  
the reason is that we want to be able write
\[ p(x)= (x -r_+)(x - r_-)\]
expanding the product above
\[ p(x) = x^2 - (r_+ + r_-)x + r_=r_-\]
so that $D$ can be written as:
\[ D = \textrm{Sign}(D)|D| \]
Observe that we can take the definition of sign as above. But then, this give a suspect idea that there can exist a \textbf{not real number} that:
\[ \pm (\sqrt{-1})^2 = -1 \]
which is also a solution of the following equation:
\[ x^2 + 1 = 0 \]
This then motivate the following concept, does arithmetic follows natural with the inclusion:
\[ z = a + b \sqrt{-1} \]
Observe the symmetry on choosing $ \sqrt{-1} $ or $ -\sqrt{-1} $. Which again motivates
\[ \bar{z} = a - b \sqrt{-1} \]
As in the textbook $ i^n$ maps to $1,i,-1,-i$ with the trick $ \mod{4 }$ 
Let us define then that addition comes naturally:
\[ (a+b\sqrt{-1}) + (c+d\sqrt{-1})= (a+d) + (b + d)\sqrt{-1}\]
Multiplication is a little bit tricky:
\[ (a + b\sqrt{-1})\times (c+d\sqrt{-1})  = (ac -bd) + (ad+bc)\sqrt{-1}\] 
which comes natural by considering $ (\sqrt{-1})^2 $.
Observe that the reals are embedded as:
\[  a + 0\sqrt{-1} \] 
The operation of taking $ z \rightarrow \bar{z} $ is called \emph{Complex conjugation}, observe that it complies $ \bar{z} \rightarrow z $, which will allow us to extract $ a,b$ as:
\begin{align*}
\frac{1}{2}(z + \bar{z}) := \textrm{Re}(z) = a
\end{align*}
and also
\begin{align*}
\frac{1}{2\sqrt{-1}}(z - \bar{z}) := \textrm{Im}(z) = b
\end{align*}
our definition above can be then replaced by:
\[ z = \textrm{Re}(z) + \textrm{Im}(z)\sqrt{-1} \]
The preceding equation proposes a big problem, since we have found another "strange" number:
\[ \frac{1}{\sqrt{-1}} = x \]
 or equivalently,
 \[ \sqrt{-1}x= 1 \]
 from which it can be solved by inspecting the table of exponentiation and we find:
 \[ x = - \sqrt{-1} \]
 In literature we often define that $ \sqrt{-1} = i$
 A big question is that if $ z \neq 0 $, then there exist $ \omega$ such that $ w \cdot z = 1 $, the answer is positive and it even has a stronger statement: with complex numbers we can solve any polynomial equation.
 \begin{mdframed}
 	\begin{align*}
 	z = a + bi \\
 	w = x + iy
 	\end{align*}
 	and we are looking $ z \cdot w = 1$
 	so we get the system of equations
 	\begin{align*}
 	ax - by =1 \\
 	bx + ay = 0
 	\end{align*}
 	and we can get information about the solution, via the determinant $ \textrm{det } = a^2 + b ^2 $, and this is only zero if only if $ a = 0 $ and $ b=0$. Solution can be found for any method
 \end{mdframed}
 Observe that if we have a complex number and its conjugate, the following relation holds:
 \begin{align*}
 z \cdot \bar{z} = a^2 + b^2 \\
 \frac{1}{z} = \frac{1}{a^2 + b^2} \cdot \bar{z}
 \end{align*}
 \begin{mdframed}
 	\begin{align*}
 	\frac{1}{2+3i}= \frac{1}{13}(2-3i)
 	\end{align*}
 	Observe that this recall Cramer's rule, since it has the form of$ \frac{1}{det} $
 \end{mdframed}
 The following problem that we are facing is the existence of:
 \[ z = a+bi \quad \exists w \quad w^2 = z \]
 if we let $ w = x + iy$, and $ w^2 = (x^2-y^2) + i (2xy)$, we get the following system of equations.
 \begin{align*}
 a = x^2 - y^2 \\
 b = 2xy
 \end{align*}
 Observe that $0$ has a unique root.
 We denote the set of all complex numbers with $ \mathbb{C}$ and observe that we can take bijection beetwen C and $R^2$
 \begin{align*}
 \mathbb{R} \leftrightarrow \mathbb{C} \\
 z \longrightarrow ( Re(z),Im(z)) \\
 a + ib \longleftarrow (a,b)
 \end{align*}
 Observe then that $ \mathbb{C} $ is a vector space over $ \mathbb{R } $. Basis: $1,i$. Now observe that we can take a geometric interpretation, given this two bijections, to the multiplication of a complex numbers.
 Side note: A linear map is a map that plays nice with arithmetic.
 an important characterization of linear maps, is that:
 \[ f(a) = a\cdot f(1) \]
 and also
 \begin{align*}
 f\left( \binom{a}{b} \right) = a \binom{1}{0} + b \binom{0}{1}
 \end{align*}
 and it can be easy characterized via:
 \begin{align*}
 \binom{v_{11} \quad v_{22}}{v_{21} \quad v_{22}}
 \end{align*}
 Now consider the following map:
\begin{align*}
\mathbb{C} \rightarrow \mathbb{C} \\
w \rightarrowtail zw
\end{align*}
where $z = a + ib$
Again observe that we are only taking care of the values on the basis.

\textbf{Example} Suppose $A^2 =  -I$, and $ \textrm{det} > 0 $, then there is an invertible matrix B such that $ A = BJB^{-1}$.
Returning to our previous point, we see that complex numbers are embedded nicely on $ 2 \times 2 $ matrices with real coefficients. So that the span $1$ and  $j$ is a 4 dimensional vector space over R. Finally, we conclude that multiplication on $\mathbb{C} $ is a matrix multiplication.

When we look a close view on $ \mathbb{R}^2 $ along with the normal dot product, we have the \textbf{Cauchy-Schwarz Inequality}:
Let us then define:
\begin{align*}
|\vec{a}|^2 = \vec{a}\cdot \vec{a}
\end{align*} 
and proof is left to the reader:
\begin{align*}
|\vec{a}\cdot \vec{b}| \leq |\vec{a}||\vec{b}|
\end{align*}