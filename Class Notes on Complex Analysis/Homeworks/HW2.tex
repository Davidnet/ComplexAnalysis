\documentclass[notitlepage]{report}
%\usepackage{lmodern}
%\usepackage[T1]{fontenc}
%\usepackage[spanish]{babel}
\usepackage[utf8]{inputenc}
\usepackage{amsmath}
\usepackage{amsfonts}
\usepackage{amssymb}
\usepackage{amsthm}
\usepackage{mdframed}
\usepackage{hyperref}
\author{David Cardozo}
\title{Cuervo es un bastardo}
\newtheorem{define}{Definition}
\newtheorem{thm}{Theorem}
\newtheorem{prop}{Proposition}
\newtheorem{lem}{Lemma}

%Customized Commands
\newcommand{\lrp}[1]{\left( #1 \right)}
\newcommand{\abs}[1]{\left| #1 \right|}
\newcommand{\set}[1]{\left\lbrace #1 \right\rbrace}
\newcommand{\RR}{\mathbb{R}}
\newcommand{\CC}{\mathbb{C}}
\newcommand{\QQ}{\mathbb{Q}}
\newcommand{\ZZ}{\mathbb{Z}}
\newcommand{\ZN}[1]{\frac{\mathbb{Z}}{#1 \mathbb{Z}}}
\newcommand{\PP}{\mathbb{P}}
\newcommand{\qt}[1]{\textrm{#1}}
\newcommand{\function}[3]{#1 : #2 \rightarrow #3}
\newcommand{\contained}{\subseteq}
\newcommand{\restric}[2]{ #1\restriction_{#2}}
\newcommand{\divs}{\mid}
\newcommand{\ndivs}{\nmid}


%%
\newcommand{\eqnumtag}{%
	% step the counter and make it lable-able
	\refstepcounter{equation}%
	% print the counter as tag
	\tag{\theequation}%
}
%%

\begin{document}
\maketitle

\item
	

\textbf{Solution}

Suppose that we are given $ P(\alpha_k) = c_k \in \CC $. In the same spirit of the above problem, we put:

\[ Q(z) = (z-\alpha_1)\cdot \ldots \cdot (z- \alpha_n) \]

Since by hypothesis we know that $ \operatorname{deg}P < n $, we can use the previous result:

\[ \frac{P(z)}{Q(z)} = \sum_{k=1}^{n} \frac{P(\alpha_k)}{Q'(\alpha_k)(z -\alpha_k)} = \sum_{k=1}^{n} \frac{c_k}{Q'(\alpha_k)(z - \alpha_k)} \]

we conclude then:

\begin{align}
 P(z) &= Q(z) \cdot \sum_{k =1}^{n} \frac{c_k}{Q'(\alpha_k)(z-\alpha_k)} \\
 &= \sum_{k=1}^{n} \frac{c_k}{Q'(\alpha_k)} \cdot \lrp{\frac{Q(z)}{z-\alpha_k}} \label{equation}
\end{align}

Now if we suppose that $ P(z) $ is given explicitly as \eqref{equation}. So that:

\[ P(\alpha_1) = \frac{c_1(\alpha_1 - \alpha_2) \ldots (\alpha_1 - \alpha_n)}{(\alpha_1 - \alpha_2) \ldots (\alpha_1 - \alpha_n)} = c_1 \]

Similarly, $ P(\alpha_k) = c_k $ for every $ k = 1, \ldots, n $. We conclude that $ P(z) $ is uniquely determined by \eqref{equation}, that is:

\[ P(z) = \sum_{k=1}^{n} c_k \prod_{j=1, j \neq k}^{n} \frac{z - \alpha_j}{\alpha_k - \alpha_j} \]
This is the famous Lagrange's interpolation polynomial.

\end{document}