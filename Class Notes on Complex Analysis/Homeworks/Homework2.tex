\documentclass[notitlepage]{report}
%\usepackage{lmodern}
%\usepackage[T1]{fontenc}
%\usepackage[spanish]{babel}
\usepackage[utf8]{inputenc}
\usepackage{amsmath}
\usepackage{amsfonts}
\usepackage{amssymb}
\usepackage{amsthm}
\usepackage{mdframed}
\usepackage{hyperref}
\author{David Cardozo}
\title{Homework 2}
\newtheorem{define}{Definition}
\newtheorem{thm}{Theorem}
\newtheorem{prop}{Proposition}
\newtheorem{lem}{Lemma}

%Customized Commands
\newcommand{\lrp}[1]{\left( #1 \right)}
\newcommand{\abs}[1]{\left| #1 \right|}
\newcommand{\set}[1]{\left\lbrace #1 \right\rbrace}
\newcommand{\RR}{\mathbb{R}}
\newcommand{\CC}{\mathbb{C}}
\newcommand{\QQ}{\mathbb{Q}}
\newcommand{\ZZ}{\mathbb{Z}}
\newcommand{\ZN}[1]{\frac{\mathbb{Z}}{#1 \mathbb{Z}}}
\newcommand{\PP}{\mathbb{P}}
\newcommand{\qt}[1]{\textrm{#1}}
\newcommand{\function}[3]{#1 : #2 \rightarrow #3}
\newcommand{\contained}{\subseteq}
\newcommand{\restric}[2]{ #1\restriction_{#2}}
\newcommand{\divs}{\mid}
\newcommand{\ndivs}{\nmid}


%%
\newcommand{\eqnumtag}{%
	% step the counter and make it lable-able
	\refstepcounter{equation}%
	% print the counter as tag
	\tag{\theequation}%
}
\usepackage{cleveref}
%%

\begin{document}
	
	\maketitle

\begin{enumerate}
	
\item
  Find the most general harmonic polynomial of the form
  \(ax^3 + bx^2y + cxy^2 + dy^3\).
  Determine the conjugate harmonic function and the corresponding analytic
  function by integration and by the formal method.
  \par\smallskip
  In order to be harmonic, \(u(x,y) = ax^3 + bx^2y + cxy^2 + dy^3\) has to
  satisfy \(\nabla^2u = 0\) so
  \[
  u_{xx} + u_{yy} = (3a + c)x + (3d + b)y = 0.
  \]
  Thus, \(3a = -c\) and \(3d = -b\) so
  \[
  u(x,y) = ax^3 - 3axy^2 - 3dx^2y + dy^3.
  \]
  To find the harmonic conjugate \(v(x,y\), we need to look at the
  Cauchy-Riemann equations.
  By the Cauchy-Riemann equations,
  \[
  u_x = 3ax^2 - 3ay^2 - 6dxy = v_y.
  \]
  Then we can integrate with respect to \(y\) to find \(v(x,y)\).
  \[
  v(x,y) = \int(3ax^2 - 3ay^2 - 6dxy)dy = 3ax^2y - ay^3 - 3dxy^2 + g(x)
  \]
  Using the second Cauchy-Riemann, we have
  \[
  v_x = 6axy - 3dy^2 + g'(x) = -u_y = 3dx^2 + 6axy - 3dy^2
  \]
  so \(g'(x) = 3dx^2\).
  Then \(g(x) = dx^3 + C\) and
  \[
  v(x,y) = 3ax^2y - ay^3 - 3dxy^2 + dx^3 + C.
  \]
\item
  Show that an analytic function cannot have a constant absolute value without
  reducing to a constant.
  \par\smallskip
  Let \(f = u(x,y) + iv(x,y)\).
  Then the modulus of \(f\) is \(\lvert f\rvert = \sqrt{u^2 + v^2}\).
  If the modulus of \(f\) is constant, then \(u^2 + v^2 = c\) for some constant
  \(c\).
  If \(c = 0\), then \(f = 0\) which is constant.
  Suppose \(c\neq 0\).
  By taking the derivative with respect to \(x\) and \(y\), we have
  \begin{align*}
    0 & = \frac{\partial}{\partial x}(u^2 + v^2)\\
      & = 2uu_x + 2vv_x\\
      & = uu_x + vv_x\\
    0 & = \frac{\partial}{\partial y}(u^2 + v^2)\\
      & = uu_y + vv_y
  \end{align*}
  Since \(f\) is analytic, \(f\) satisfies the Cauchy-Riemann.
  That is, \(u_x = v_y\) and \(u_y = -v_x\).
  \begin{subequations}
    \begin{align}
      uv_y + vv_x & = 0\label{2.1.2.4a}\\
      -uv_x + vv_y & = 0\label{2.1.2.4b}
    \end{align}
  \end{subequations}
  Setting \cref{2.1.2.4a} equal to \cref{2.1.2.4b}, we have
  \[
  v_x(u + v) + v_y(u - v) = 0.
  \]
  Now, either \(v_x\) and \(v_y\) are zero, \(v_x\) and \(u - v\) are zero,
  \(v_y\) and \(u + v\) are zero, or \(u + v\) and \(u - v\) are zero.
  If \(v_x = v_y = 0\), then \(f\) is constant.
  If \(v_x = 0\) and \(u - v = 0\), then \(u_y = 0\) and \(u = v\).
  Since \(u = v\) and \(v_x = 0\), then so does \(u_x = 0\) and it also follows
  that \(v_y = 0\); thus, \(f\) is a constant.
  By the same argument, \(f\) is a constant when \(v_y = 0\) and \(u + v = 0\).
  If \(u + v = 0\) and \(u - v = 0\), then \(u = \pm v\) so \(u = v = 0\) and
  \(f\) is a constant.
\item
  Prove rigorously that the functions \(f(z)\) and \(\overline{f(\bar{z})}\)
  are simultaneously analytic.
  \par\smallskip
  Let \(g(z) = \overline{f(\bar{z})}\) and suppose \(f\) is analytic.
  Then \(g'(z)\)  is
  \begin{align*}
    g'(z) & = \lim_{\Delta z\to 0}\frac{g(z + \Delta z) - g(z)}{\Delta z}\\
          & = \lim_{\Delta z\to 0}
            \frac{\overline{f(\bar{z} + \overline{\Delta z})} -
            \overline{f(\bar{z})}}{\Delta z}\\
          & = \lim_{\Delta z\to 0}\biggl[
            \overline{\frac{f(\bar{z} + \overline{\Delta z}) - f(\bar{z})}
            {\overline{\Delta z}}}\biggr]\\
    \intertext{Since conjugation is continuous, we can move the limit inside
    the conjugation.}
          & = \overline{\lim_{\Delta z\to 0}
            \frac{f(\bar{z} + \overline{\Delta z}) - f(\bar{z})}
            {\overline{\Delta z}}}\\
          & = \overline{f'(\bar{z})}
  \end{align*}
  Thus, \(g\) is differentiable with derivative \(\overline{f'(\bar{z})}\).
  Suppose \(\overline{f(\bar{z})}\) is analytic and let
  \(\overline{g(\bar{z})} = f(z)\).
  Then by the same argument, \(f\) is differentiable with derivative
  \(\overline{g'(\bar{z})}\).
  Therefore, \(f(z)\) and \(\overline{f(\bar{z})}\) are simultaneously
  analytic.
  \par\smallskip
  We could also use the Cauchy-Riemann equations.
  Let \(f(z) = u(x,y) + iv(x,y)\) where \(z = x + iy\) so \(\bar{z} = x - iy\).
  Then \(\overline{f(\bar{z})} = \alpha(x,y) - i\beta(x,y)\) where
  \(\alpha(x,y) = u(x,-y)\) and \(\beta(x,y) = v(x,-y)\).
  In order for both to be analytic, they both need to satisfy the
  Cauchy-Riemann equations.
  That is, \(u_x = v_y\), \(u_y = -v_x\), \(\alpha_x = \beta_y\) and
  \(\alpha_y = -\beta_x\).
  \begin{align*}
    u_x(x,y) & = v_y(x,y)\\
    u_y(x,y) & = -v_x(x,y)\\
    \alpha_x(x,y) & = u_x(x,-y)\\
    \alpha_y(x,y) & = -u_y(x,-y)\\
    -\beta_x(x,y) & = v_x(x,-y)\\
    \beta_y(x,y) & = v_y(x,-y)
  \end{align*}
  Suppose that \(\overline{f(\bar{z})}\) satisfies the Cauchy-Riemann
  equations.
  Then \(\alpha_x = u_x(x,-y) = v_y(x,-y) = \beta_y\) and
  \(\alpha_y = -u_y(x,-y) = v_x(x,-y) = -\beta_x\).
  Therefore,
  \begin{align*}
    u_x(x,-y) & = v_y(x,-y)\\
    u_y(x,-y) & = -v_x(x,-y)
  \end{align*}
  which means \(f(\bar{z})\) satisfies the Cauchy-Riemann equations.
  Now, recall that \(\lvert z\rvert = \lvert\bar{z}\rvert\).
  Since \(f(\bar{z})\) satisfies the Cauchy-Riemann equations, for an
  \(\epsilon > 0\) there exists a \(\delta > 0\) such that when 
  \(0 < \lvert\Delta z\rvert < \delta\),
  \(\lvert f(\bar{z}) - \bar{z}_0\rvert = \lvert f(z) - z_0\rvert < \epsilon\).
  Thus, \(\lim_{\Delta z\to 0}f(z) = z_0\) so \(f(z)\) is analytic if
  \(\overline{f(\bar{z})}\) is analytic.
\item
  Prove that the functions \(u(z)\) and \(u(\bar{z})\) are simultaneously
  harmonic.
  \par\smallskip
  Since \(u\) is the real part of \(f(z)\), \(u(z) = u(x,y)\) where
  \(z = x + iy\).
  Suppose \(u(z)\) is harmonic.
  Then \(u(z)\) satisfies Laplace equation.
  \[
  \nabla^2u(z) = u_{xx} + u_{yy} = 0
  \]
  Now, \(u(\bar{z}) = u(x,-y)\) where
  \(\frac{\partial^2}{\partial x^2}u(\bar{z}) = u_{xx}\) and
  \(\frac{\partial^2}{\partial y^2}u(\bar{z}) = u_{yy}\) so
  \[
  \nabla^2u(\bar{z}) = u_{xx} + u_{yy} = 0.
  \]
  Since \(u(z)\) is harmonic, \(u_{xx} + u_{yy} = 0\) so it follows that
  \(u(\bar{z})\) is harmonic as well.
  
  \item
  If \(Q\) is a polynomial with distinct roots \(\alpha_1,\ldots,\alpha_n\),
  and if \(P\) is a polynomial of degree \(< n\), show that
  \[
  \frac{P(z)}{Q(z)} = \sum_{k = 1}^n
  \frac{P(\alpha_k)}{Q'(\alpha_k)(z - \alpha_k)}.
  \]
  Let's multiple by \(Q(z)\).
  We then have
  \[
  P(z) = \sum_{k = 1}^n\frac{P(\alpha_k)}{Q'(\alpha_k)(z - \alpha_k)}Q(z)
  \]
  which are both polynomials of degree less than \(n\) and agreeing at
  \(z = \alpha_k\).
  
  \item 
	  Use the formula in the preceding exercise to prove that there exists a
	  unique polynomial \(P\) or degree \(< n\) with given values \(c_k\) at the
	  points \(\alpha_k\) (Lagrange's interpolation polynomial).
 
 Suppose that we are given $ P(\alpha_k) = c_k \in \CC $. In the same spirit of the above problem, we put:
 
 \[ Q(z) = (z-\alpha_1)\cdot \ldots \cdot (z- \alpha_n) \]
 
 Since by hypothesis we know that $ \operatorname{deg}P < n $, we can use the previous result:
 
 \[ \frac{P(z)}{Q(z)} = \sum_{k=1}^{n} \frac{P(\alpha_k)}{Q'(\alpha_k)(z -\alpha_k)} = \sum_{k=1}^{n} \frac{c_k}{Q'(\alpha_k)(z - \alpha_k)} \]
 
 we conclude then:
 
 \begin{align}
 P(z) &= Q(z) \cdot \sum_{k =1}^{n} \frac{c_k}{Q'(\alpha_k)(z-\alpha_k)} \\
 &= \sum_{k=1}^{n} \frac{c_k}{Q'(\alpha_k)} \cdot \lrp{\frac{Q(z)}{z-\alpha_k}} \label{equation}
 \end{align}
 
 Now if we suppose that $ P(z) $ is given explicitly as \eqref{equation}. So that:
 
 \[ P(\alpha_1) = \frac{c_1(\alpha_1 - \alpha_2) \ldots (\alpha_1 - \alpha_n)}{(\alpha_1 - \alpha_2) \ldots (\alpha_1 - \alpha_n)} = c_1 \]
 
 Similarly, $ P(\alpha_k) = c_k $ for every $ k = 1, \ldots, n $. We conclude that $ P(z) $ is uniquely determined by \eqref{equation}, that is:
 
 \[ P(z) = \sum_{k=1}^{n} c_k \prod_{j=1, j \neq k}^{n} \frac{z - \alpha_j}{\alpha_k - \alpha_j} \]
 This is the famous Lagrange's interpolation polynomial.
 
\end{enumerate}

\end{document}