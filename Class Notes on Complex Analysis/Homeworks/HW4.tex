\documentclass[notitlepage]{article}
%\usepackage{lmodern}
%\usepackage[T1]{fontenc}
%\usepackage[spanish]{babel}
\usepackage[utf8]{inputenc}
\usepackage{amsmath}
\usepackage{amsfonts}
\usepackage{amssymb}
\usepackage{amsthm}
\author{David Cardozo}
\title{Complex Analysis}

\newtheorem{thm}{Theorem}
\newtheorem{lem}[thm]{Lemma}
\newtheorem{prop}{Proposition}[section]

\theoremstyle{definition}
\newtheorem*{define}{Definition}
\newtheorem{defn}{Definition}[section]
\newtheorem{conj}{Conjecture}[section]
\newtheorem{exm}{Example}[section]
\theoremstyle{remark}
\newtheorem*{rem}{Remark}
\newtheorem*{note}{Note}
\newtheorem{case}{Case}
\newtheorem{exc}{Exercise}
\newtheorem*{sol}{Solution}









%For Chapter 1
\usepackage{mdframed} %frames around examples 

%Customized Commands
\newcommand{\lrp}[1]{\left( #1 \right)}
\newcommand{\abs}[1]{\left| #1 \right|}
\newcommand{\set}[1]{\left\lbrace #1 \right\rbrace}
\newcommand{\RR}{\mathbb{R}}
\newcommand{\CC}{\mathbb{C}}
\newcommand{\QQ}{\mathbb{Q}}
\newcommand{\ZZ}{\mathbb{Z}}
\newcommand{\ZN}[1]{\frac{\mathbb{Z}}{#1 \mathbb{Z}}}
\newcommand{\PP}{\mathbb{P}}
\newcommand{\qt}[1]{\textrm{#1}}
\newcommand{\function}[3]{#1 : #2 \rightarrow #3}
\newcommand{\contained}{\subset}
\newcommand{\restric}[2]{ #1\restriction_{#2}}
\newcommand{\divs}{\mid}
\newcommand{\ndivs}{\nmid}
\newcommand{\gothic}[1]{\mathfrak{#1}}
\newcommand{\inv}{^{-1}}

\begin{document}
\maketitle
\begin{exc}
	If $f(z)$ and $g(z)$ have the algebraic orders $h$ and $k$ at $z=a$, show that:
	\begin{itemize}
	\item $fg$ has the order $h+k$
	\item $f/g$ the order $h-k$
	\item  $f+g$ an order which does not exceed $\max\ (h,k)$.
	\end{itemize}
\end{exc}
\begin{sol}
	\begin{itemize}
		
		1. To say that $f(z)$ and $g(z)$ have algebraic orders $h$ and $k$ respectively at $z = a$ is to say that
		
		$$
		\lim_{n \to \infty} |(z-a)|^{\alpha} |f(z)| = 0 \text{ for all } \alpha > r
		$$
		
		and
		
		$$
		\lim_{n \to \infty} |(z-a)|^{\beta} |g(z)| = 0 \text{ for all } \beta > k
		$$
		
		where $h$ and $k$ are the minimal integers satisfying these properties.
		
		2. Then we have that
		
		\begin{align*}
		& \lim_{n \to \infty} |(z-a)|^{\alpha+\beta} |f(z) \cdot g(z)| = 0 \\
		\iff & \lim_{n \to \infty} |(z-a)|^{\alpha} |(z-a)|^{\beta} |f(z) \cdot g(z)| = 0\\
		\iff & \alpha + \beta > h + k \\
		\end{align*}
		
		as desired.
		\item First, we observe that if $g(z) \ne 0$, then
		
		\begin{align*}
		& \lim_{n \to \infty}  |(z-a)|^{\beta} \left|g(z)\right| = 0 \\
		\iff & \lim_{n \to \infty} |(z-a)|^{-\beta} \left|{1 \over g(z)}\right| = 0 \\
		\iff & - \beta > - k \\
		\end{align*}
		
		2. Then if we continue to assume that $g(z) \ne 0$, we have that
		
		\begin{align*}
		& \lim_{n \to \infty} |(z-a)|^{\alpha-\beta} \left|{f(z) \over g(z)}\right| = 0 \\
		\iff & \lim_{n \to \infty} |(z-a)|^{\alpha} |(z-a)|^{-\beta} \left|{f(z) \over g(z)}\right| = 0\\
		\iff & \alpha - \beta > h - k \\
		\end{align*}
		
		as desired.
		\item 
		If $f(z) = z^n f'(z)$, and $g(z) = z^n g'(z)$, then $f(z)+g(z) = x^n (f'(z)+g'(z))$, which means that if both orders are at least $n$, then the sum is at least $n$.  In other words, the order of the sum must be at least the minimum of the two orders.
		
	\end{itemize}
\end{sol}


\begin{exc}
	Show that a function which is analytic in the whole plane and has a nonessential singularity at $ \infty $ reduces to a polynomial.
\end{exc}
\begin{sol}
	Let $ f $ be a function that is analytic in the whole plane and has a nonessential singularity at $ \infty $, define $ F(z) = f(\frac{1}{z}) $. Then $ F $ has a nonessential singularity at $ z = 0 $, therefore we have two cases:
	
	\begin{itemize}
		\item \textit{Case I} The singularity is removable. If the singularity is removable then $ F $ is a bounded function in a neighborhood of of zero and $ f $ is bounded at infinity, and with the analyticity of $ f $ we have then, that it is a constant.
		\item \textit{Case II} The singularity is a pole. Suppose that the singularity is a pole, then:
		\[ F(z) = f(\frac{1}{z}) = \sum_{k = 1}^{n} c_k z^k + g(z) \]
			and $ g $ is analytic at $ 0 $. Which then can use as:
			\[ f(z) = g\lrp{\frac{1}{z}} = \sum_{k = 1 }^{n}c_k z^k + g(\frac{1}{z}) \]
			we observe that $ g(\frac{1}{z}) $ is bounded in a neighborhood of zero since it is $ f $ and the rest of a polynomial, and $ g(\frac{1}{z}) $ then is analytic on the entire complex plane and has a finite limit $ g(0) $, which means that is just a constant.
	\end{itemize}

\end{sol}
\begin{exc}
	Show that the functions $ e^z, \sin(z), $ and $ \cos(z) $ have essential singularities at $ \infty $
\end{exc}
\begin{sol}
	We observe that:
	\[ \lim\limits_{z \rightarrow 0^+} \abs{e^{1/z}} = \infty \]
	and 
	\[ \lim\limits_{z \rightarrow 0^-} \abs{e^{1/z}} = 0 \]
	so that 
	\[ 0 \neq \lim\limits_{z \rightarrow 0} \abs{z}^{\alpha}\abs{e^{1/z}} \neq \infty \]
	so we observe that $ e^z $ has an essential singularity at $ \infty $.
	
	We observe that $ \cos(z) $ has essential singularity at $ \infty $ iff $ \cos(\frac{1}{z}) $ has essential singularity $ 0 $. Now, we proceed by contradiction. Suppose that $\cos (1/z)$   has a pole or removable singularity at $0$. Then the same is true for its derivative $z^{-2}\sin (1/z)$.	Since 
	$$e^{i/z}=\cos (1/z)+i\sin(1/z)$$
	we have a contradiction: essential singularity on the left but not on the right.
	
	Argument is similar for $ \sin(z) $ 
\end{sol}



\begin{exc}
	Show that any function which is meromorphic in the extended plane is rational
\end{exc}
\begin{sol}
	Our main idea of thought will be to prove the following:
	\begin{thm}
		Suppose $ f: C_\infty \rightarrow C_\infty $ is a meromorphic function in the extended complex plane. Then $ f $ is a rational function.
	\end{thm}
	\begin{proof}
		Let $ \set{z_n} \in \CC $ be the set of the poles of the function $ f $. The function $ F(z) = f(\frac{1}{z}) $ must be analytic in a deleted neighborhood of the origin, hence $ f $ is analytic in a deleted neighborhood of $ \infty $, the rest of the complex plane can contain only finitely many singularities, which implies $ \set{z_n} $ is finite. Now suppose that the orders of $ z_1, \ldots, z_n $ with multiplicities $ m_1, \ldots, m_k $ and let $ b_1, \ldots, b_l $ the poles of $ f $ with orders $ o_1, \ldots, o_l $. Now consider:
		\[ g(z) = \frac{\prod_{j =1}^{k}(z- z_j)^{m_j}}{\prod_{d=1}^{l}(z - b_d)^{o_d} }\]
		Observe $ g $ has exactly the same zeros and poles of $ f $ with the same multiplicities, so $ h: \CC_\infty \rightarrow \CC_\infty $ defined by:
		\[ h(z) = \frac{f(z)}{g(z)} \]
		is meromorphic function with no zeros or poles, then $ h $ extends to a nonzero bounded entire function, so by Liouville, $ h(z) = c $ for $ c \neq 0 $. Then 
		\[ f(z) = c g(z) = c \frac{\prod_{j =1}^{k}(z- z_j)^{m_j}}{\prod_{d=1}^{l}(z - b_d)^{o_d} } \]
		Then $ f $ is rational function
	\end{proof}
\end{sol}
\textbf{\large Problem 2}
Find zeroes and orders of zeroes of the following functions
\begin{enumerate}
	\item $ \frac{z^2 +1}{z^2 - 1} $
	
	The zeros are $ i $ and $ -i $. Which both have order $ 1 $
	\item $ \frac{z^4 + 1}{z^5} $
	
	The zeros are $ -\sqrt{i}, \sqrt{i}, \sqrt{-i}, - \sqrt{-i}, \infty $. Which all have order $ 1 $. For the infinite case, consider $ F(z) = f(1/z)  = z + z^5 $ so that it has order 1.
	\item $ z^2 \sin(z) $
	We see that the zeros are $ n\pi n \in \ZZ $, and the order of zeros different from 0 is 1, lastly, the order of 0 is 2.
	\item $ \cos(z) -1 $
	We see that the zeros of the function are of the form $ 2 \pi n, n \in \ZZ $ and each of them are of order 2
	\item $ \frac{\cos(z) -1 }{z} $
	
	By periodicity, the zeros are $ 2 \pi n $ and each have order $ 2 $
	\item $ \frac{\cos(z) -1 }{z^2} $
	
	Again the zeros are $ 2 \pi n $ and each have order $ 2 $
	
	\item $ e^z - 1 $
	From periodicity of $ e^z $ the zeros are of the form $ 2 \pi n $, and each have order $ 1 $
\end{enumerate}

\textbf{\large Problem 2}

\begin{itemize}
	\item Which of the functions in porblem $ 1 $ are holomorphic at $ \infty $ ?
	
	Since we are looking for essential singularities, the only functions holomorphic at $ \infty $ are $ 1 $ and $ 2 $ becuase are regular in $ \infty $. The rest have essential singularities at  $ \infty $
	
	\item For functions in problem $ 1 $ which are holomorphic at $ \infty $ determine the order of any zeros at $ \infty $
	
	For (2), the order of $ \infty $ is of order 1.
\end{itemize}
\end{document}



