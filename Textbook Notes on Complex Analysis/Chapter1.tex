\chapter{Complex Numbers}
\section{The Algebra of Complex Numbers}\label{sec:the-algebra-of-complex-numbers}
We define the imaginary unit as $ i^2 = -1 $, and given two real numbers $ \alpha$ and $ \beta$ we can get the following complex number $ \alpha + i \beta$,  we define the \textbf{real} part of a complex number as: $ \re{ z } = \alpha $, and the \textbf{imaginary} part of a complex number as $ \im{z} = \beta $. If $ \alpha = 0$ the number is said to be \emph{purely imaginary}; if $ \beta = 0$ it is \emph{real}. Observe then, that zero is the only number which is at once real and purely imaginary. Two complex numbers are equal if and only if have the same real part and the same imaginary part.
We define addition and multiplication as follows:
 
\begin{align} \label{Addition}
(\alpha + i \beta) + ( \gamma + i \delta) = (\alpha + \gamma) + i(\beta + \gamma) 
\end{align}

and

\begin{align} \label{multiplication}
(\alpha + i \beta) ( \gamma + i \delta) = ( \alpha \gamma - \beta \delta) + i ( \alpha \delta + \beta \gamma)
\end{align}

Observe that for \eqref{multiplication} we have used the property that $ i ^2 = -1 $

It is less obvious if the operation of division is possible:

\begin{thm}
	\textbf{Existence of Division} 
	$\frac{ \cexo }{ \cext }$ is a complex number, provided that $ \gamma + i \delta \neq 0 $
\end{thm}
\begin{proof}
	Let us denote $ x + i $ the quotient result from the division, for that to happen we must have:
	\begin{align*}
	\cexo = (\cext)(x + iy) 
	\end{align*}
	By \eqref{multiplication} this is equivalent to:
	\begin{align*}
	\cexo = (\gamma x - \delta y ) + i ( \delta x + \gamma y)
	\end{align*}
	And thus, we obtain the equations:
	\begin{align*}
	\alpha = \gamma x - \delta y \\
	\beta =  \delta x + \gamma y
	\end{align*}
	Since this a $ 2 \times 2$ system of equation, it has the unique solution:
	\begin{align*}
	x = \frac{\alpha \gamma  + \beta \delta}{\gamma^2 + \delta^2} \\
	y = \frac{ \beta \gamma - \alpha \delta}{\gamma^2 + \delta^2}
	\end{align*}
	Provided $ \gamma^2 + \delta^2 $ is not zero.
	We have thus the result:
	\begin{align*}
	\frac{ \cexo }{ \cext } =  \frac{\alpha \gamma  + \beta \delta}{\gamma^2 + \delta^2} + i \frac{ \beta \gamma - \alpha \delta}{\gamma^2 + \delta^2} 
	\end{align*}
	Which as for inspection, can be seen as a complex number.
\end{proof}
Once proved that the quotient is a complex number, it is rather cumbersome to compute the quotient in the previous manner, so that we use the following method (Once the existence of the operation is proven):
\begin{align*}
	\frac{\cexo }{\cext  } = \frac{(\cexo )(\cextc )}{(\cext )(\cextc )} = \frac{(\alpha \gamma + \beta) + i(\beta \gamma - \alpha \delta)}{\gamma^2 + \delta^2}
\end{align*} 
Observe then that the \textbf{reciprocal} of a complex number is:
\begin{align*}
\frac{1}{ \alpha + i \beta} = \frac{ \alpha - i \beta}{\alpha^2 + \beta^2}
\end{align*}
We also take the observation that $ i^n $ can only have the four possible values: $1, i, -1, -i$. They correspond to values of $n$ which divided by $4$ leave the remainders $ 0,1,2,3 $.

\begin{mdframed}
	Calculate $i^{-18}$.
	
	\textit{Solution}: Observe that:
	\begin{align*}
	i^{-18} = (i^{18})^{-1}
	\end{align*}
	and then $ x \equiv 18 \mod{4} $ is $ x \equiv 2 \mod{4} $ so that 
	\[ i^{18} = -1 \]
	and finally we have:
	\[ i^{-18} = (-1)^{-1} = -1 \]
	Which is the correct answer.
	
\end{mdframed}
\subsection{Exercises}
\subsubsection{1.}
Find the values of:
\begin{align*}
(1 + 2i)^{3} &= \binom{3}{0}1^3 + \binom{3}{1}1^2 2i + \binom{3}{2}(2i)^2 + \binom{3}{3} (2i)^3 \\
& = 1 + 3 \cdot 2i + 3 \cdot 2^2 i^2 + 1 \cdot 2^3 i^3 \\
& = 1 + 6i - 12 - 8i \\
& = -11 - 2i 
\end{align*}
\rule{\textwidth}{1pt}
\begin{align*}
\frac{5}{-3 + 4i} &= \frac{5}{-3 + 4i} \frac{-3-4i}{-3-4i} \\
 &= \frac{-15 - 20i}{(-3)^{2} -(4i)^{2}} \\
 &= \frac{5 (-3 - 4i)}{5 \cdot 5} \\
 &= \frac{-3-4i}{5} 
\end{align*}
\rule{\textwidth}{1pt}
\begin{align*}
(\frac{2+i}{3-2i})^{2} &= (\frac{2+i}{3-2i} \cdot \frac{3 + 2i}{3 +2i})^{2} \\
 &= (\frac{4 + 7i}{9 + 4 })^2 \\
 &= \frac{56i - 33}{169} 
\end{align*}
\rule{\textwidth}{1pt}
\begin{align*}
(1+i)^n + (1-i)^n &= \sqrt{2}^n e^{\frac{\pi}{4}in} + \sqrt{2}^n e^{-\frac{\pi}{4}in} \\
 & = \sqrt{2}^n(\cos(\frac{\pi}{4}n) + i \sin(\frac{\pi}{4}n) + \cos(-\frac{\pi}{4}n) + i \sin(-\frac{\pi}{4}n)) \\
 & = 2^{\frac{n}{2} + 1} \cos(n \frac{\pi}{4})
\end{align*}
\subsubsection{2.}
If $ z = x + iy $, find the real and imaginary parts of:
Observe that:
\begin{align*}
	z^4 &= (x + iy)^4 \\
	 &= \binom{4}{0}x^4 + \binom{4}{1}x^3iy + \binom{4}{2} x^2(iy)^2 + \binom{4}{3}x(iy)^3 + \binom{4}{4} (iy)^4 \\
	 &= x^4 + 4 x^3iy + (-6) x^2 y^2 + 4 xy^3(-i) + y^4 \\
	 &= x^4 -6x^2y^2 + y^4 + i(4x^3y - 4xy^3)
\end{align*}
And we explicitly see that:
\begin{align*}
\re{z^4} = x^4 -6x^2y^2 + y^4 \quad \im{z^4} = (4x^3y - 4xy^3)
\end{align*}
\rule{\textwidth}{1pt}
We recall the form of $ \frac{1}{z} $ is:
\begin{align*}
\frac{1}{z} = \frac{ x - i y}{x^2 + y^2}
\end{align*}
And we explicitly see that:
\begin{align*}
\re{ \frac{1}{z} } = \frac{ x }{x^2 + y^2} \quad \im{\frac{1}{z}} = \frac{  - y}{x^2 + y^2}
\end{align*}
\rule{\textwidth}{1pt}
Let us take note that
\begin{align*}
\frac{z-1}{z+1} &= \frac{(x -1) + iy}{(x + 1) + iy} \\
&= \frac{((x-1)+iy)((x+1)-iy)}{(x+1)^2 + y^2} \\
&= \frac{(x^2 - 1) + y^2}{(x+1)^2 + y^2} + i\frac{(x+1)y - (x-1)y}{(x+1)^2 + y^2} \\
&= \frac{(x^2 - 1) + y^2}{(x+1)^2 + y^2} + i\frac{2y}{(x+1)^2 + y^2}
\end{align*}
And we explicitly see that:
\begin{align*}
\re{ \frac{z-1}{z+1} } &= \frac{(x^2 - 1) + y^2}{(x+1)^2 + y^2} \\ 
\im{ \frac{z-1}{z+1} } &= \frac{(x+1)y - (x-1)y}{(x+1)^2 + y^2}
\end{align*}
\rule{\textwidth}{1pt}
\begin{align*}
\frac{1}{z^2} &= \frac{1}{(x + iy)^2} \\
 &= \frac{1}{(x^2 - y^2) + 2xyi} \\
 &= \frac{(x^2 - y^2 ) -2xyi}{(x^2 - y^2)^2 - 4x^2y^2}
\end{align*}
So that, we observe:
\begin{align*}
\re{ \frac{1}{z^2} } &= \frac{(x^2 - y^2 )}{(x^2 - y^2)^2 - 4x^2y^2} \\ 
\im{ \frac{1}{z^2} } &= \frac{-2xy}{(x^2 - y^2)^2 - 4x^2y^2}
\end{align*}
\subsubsection{3.}
Show that:
\begin{align*}
(\frac{-1 \pm i \sqrt{3}}{2})^3 = 1
\end{align*}
First, let us observe that:
\begin{align*}
(\frac{-1 \pm i \sqrt{3}}{2})^3 &= \left( \frac{ (2 e^{\pm \frac{\pi}{3}i})}{2} \right)^3 \\
 &= e^{ \pm \pi i} \\
 &= \cos(\pi) + i \sin(\pm \pi) \\
 &= \cos(\pi) \\
 &= 1
\end{align*}
\rule{\textwidth}{1pt}
Consider:
\begin{align*}
\frac{\pm 1 \pm i \sqrt{3}}{2} &= e^{\pm \frac{\pi}{3}} \\
&\text{or} \quad e^{\pm \frac{4 \pi}{3}}
\end{align*}
We consider then two cases:
\begin{align*}
	\left( \frac{\pm 1 \pm i \sqrt{3}}{2} \right)^6  &= \cos( \pm \frac{6 \pi}{3} ) + i \sin( \pm \frac{6 \pi}{3}) \\
	&= \cos(\pm 2 \pi) + i \sin( \pm 2 \pi) \\
	&= \cos( 2 \pi) \\
	&= 1
\end{align*}
Finally,
\begin{align*}
\left( \frac{ \pm 1 \pm i \sqrt{3} }{2} \right) &= \cos( \pm \frac{24 \pi}{3}) + i \sin( \pm \frac{24 \pi}{3}) \\
&= \cos(8 \pi) + i \sin(8 \pi) \\
&= 1
\end{align*}
Which proves the equality.
\section{Square Roots}
The main idea is that we are looking for a number $ x + iy$ such that
\[ 
(x + iy)^{2} = \cexo 
 \]
 Evidently this is same as:
 \begin{align*}
 x^2 - y^2 = \alpha \\
 2xy = \beta
 \end{align*}
Let us take note that
\[ 
(x^{2}+y^{2})^{2} = (x^2 + y^2)^2 + 4x^2y^2
\]
and combining this information along with the previous relationships:
\[ (x^{2}+y^{2})^{2} = \alpha^2 + \beta^2 \]
Hence we should have:
\[ x^2 + y^2 = \sqrt{ \alpha^2 + \beta^2 } \]
and using again the two equalities stated above:
\begin{align*}
x^2 = \frac{1}{2}( \alpha + \sqrt{\alpha^2 + \beta^2}) \\
y^2 = \frac{1}{2}( -\alpha + \sqrt{ \alpha^2 + \beta^2}  )
\end{align*}
Take note, that as much, these two quantities are as much, zero or positive (take extreme values of $\beta$ )
A word of caution the previous equation should be used with caution, since we must be careful to select $x$ and $y$ so that their product has the sign of $\beta$.
In general, but much more cumbersome, a general solution is:
 \begin{align*}
  \sqrt{\cexo }= \pm \left( \sqrt{ \frac{ \alpha + \sqrt{\alpha^2 + \beta^2} }{2}} + i \frac{\beta}{\left| \beta \right|} \sqrt{ \frac{ -\alpha + \sqrt{\alpha^2 + \beta^2}}{2}} \right)
 \end{align*}
 Provided that $ \beta \neq 0 $. As from elementary algebra, we see that for $ \beta = 0$ the values are $ \pm \sqrt{\alpha} $ if $ \alpha \geq 0$, $\pm i \sqrt{-\alpha} $ if $ \alpha < 0$. It is understood that all square rots of positive numbers are taken with the positive sign.
 Observe then that the square root of any complex number exist.
 
 \subsection{Exercises}
\subsubsection{1.}
Compute $\sqrt{i}$, let us observe
\begin{align*}
(x + iy)^2 &= 0 + i \\
x^2 + 2xiy - y^2 = 0 + i
\end{align*}
Which give us the following system:
\begin{align*}
x^2 - y^2 = 0 \\
2xy = 1
\end{align*}
So that 
\begin{align*}
(x^2 + y^2)^2 &= (x^2 - y^2)^2 + 4x^2y^2 \\
&= 1 \\
\end{align*}
so that we have
\begin{align*}
	x^2 &= \frac{1}{2} \\
	y^2 &= \frac{1}{2}
\end{align*}
so that we must set up the possible values of $x$ and $y$ so that $ \beta $ has the right sign.
\begin{align*}
z_1 &= \frac{1}{\sqrt{2}} + i\frac{1}{\sqrt{2}} \\
z_2 &= -\frac{1}{\sqrt{2}} - i\frac{1}{\sqrt{2}}
\end{align*}
\rule{\textwidth}{1pt}
Let us find $x,y$ so that:
\begin{align*}
(x + iy)^2 &= -i \\
x^2 + 2xiy - y^2 = -i
\end{align*}
so that
\begin{align*}
x^2 - y^2 &= -i \\
2xy &= -1 \\
x^2 + y^2 &= 1  
\end{align*}
and we get:
\begin{align*}
x^2 = \frac{1}{2} \\
y^2 = \frac{1}{2}
\end{align*}
and since we need $2xy = -1$, we see that the roots are:
\begin{align*}
z_1 &= \frac{1}{\sqrt{2}} - \frac{1}{\sqrt{2}}i \\
z_2 &= -\frac{1}{\sqrt{2}} + \frac{1}{\sqrt{2}}i
\end{align*}
\rule{\textwidth}{1pt}
\begin{align*}
(x + iy)^2 &= 1 + i \\
x^2 + 2xiy - y^2 &= 1 + i\\
\end{align*}
and we get the system of equalities:
\begin{align*}
x^2 - y^2 = 1 \\ 
2xy = 1
\end{align*}
and after some algebra:
\begin{align*}
x^2 &= \frac{\sqrt{2} + 1}{2} \\
y^2 &= \frac{-1 + \sqrt{2} }{2}
\end{align*}
and having the condition of $x \cdot y > 0 $, we conclude:
\begin{align*}
z_1 &= \sqrt{\frac{\sqrt{2} + 1}{2}} + i \sqrt{\frac{-1 + \sqrt{2}}{2} }\\
z_2 &= -\sqrt{\frac{(\sqrt{2} + 1)}{2}} - i \sqrt{ \frac{-1 + \sqrt{2} }{2} }
\end{align*}
\rule{\textwidth}{1pt}
We want to find $x,y$ so that:
\begin{align*}
(x + iy)^2 = \frac{1 - i \sqrt{3}}{2}
\end{align*}
which give us the system of equalities
\begin{align*}
\begin{cases}
x^2 - y^2 &= \frac{1}{2} \\
2xy &= -\frac{\sqrt{3}}{2} 
\end{cases}
\end{align*}
which ends up with the specification of: 
\begin{align*}
	x^2 &= \frac{3}{4} \\
	y^2 &= \frac{1}{4}
\end{align*}
So that we found:
\begin{align*}
z_1 &= \frac{\sqrt{3}}{2} - i\frac{1}{2} \\
z_2 &= -\frac{\sqrt{3}}{2} + i \frac{1}{2}
\end{align*}
\subsubsection{2.} Find the four values of $ \sqrt[4]{-1} $

We begin then with looking solutions of the form:
\begin{align*}
(x + iy)^4	==ssds
\end{align*} 

 
 \section{Justification}
 We recall that $ \RR $ is a \emph{field}. This is equivalently that the operations of addition and multiplication complies with the associative, commutative, and distributive laws, The numbers $0$ and $1$ are neutral elements of addition and multiplication respectively. Moreover, the equation of subtraction $ \beta + x = \alpha $ has always a solution, and the equation of division $ \beta x = \alpha $ has a solution, whenever it makes sense. It is a routine exercise to prove that the neutral elements are unique. Also, every field is an \emph{integral domain }, i.e, $ \alpha \beta = 0 $, if and only if $ \alpha = 0$ and $ \beta = 0 $. These properties are common to all fields.
 
 In addition $\RR$ has an \emph{order relation} $ \alpha < \beta $, which is defined as $ \alpha < \beta $ if and only if $ \beta - \alpha \in \mathbb{R}^+ $. Observe that the set $ \RR^+ $ is characterized via: (1) 0 is not a positive number; (2) if $ \A \neq 0 $ either $ \A $ or $ -\alpha $ is positive, (3) the sum and the product of two positive numbers are positive. From these three condition one derives all the usual rules for manipulating inequalities. It is left to the reader to prove that every square $ \alpha^2 $ is either positive or zero. Observe that because of the order relation $1, 1+1, 1+1+1, \ldots $ are all different. Hence $ \RR $ contains the natural numbers, and since it is a field it must contain the subfield formed by all rational numbers.
 
 Finally, $ \RR$ satisfies the \emph{completeness condition}, i.e, every increasing and bounded sequence of real numbers has a limit.
 
 Take note that the equation $ x^2 + 1 =0 $ has no solution in $ \RR $ since $ x^2 + 1 $ is always positive.
 
\section*{Triangle Inequality and Important stuff}
We have the following inequalities
\begin{align*}
\abs{x - z} \leq \abs{x - y} + \abs{y - z} \\
\end{align*} 
which has the interpretation of going directly is shorter than going to intermediate points.
\begin{align*}
\abs{x + y} \leq \abs{x} + \abs{y}
\end{align*}
which is the usual geometric interpretation of geometry triangle.

Finally, we have the interesting reverse triangle inequality.

\[ \abs{\abs{x} - \abs{y}} \leq \abs{x-y} \]

which is proven as follows:

\textit{Proof} Observe:
\begin{align*}
\abs{(z-w)+w} &\leq \abs{z-w} + \abs{w} \\
\abs{z} &\leq \abs{z-w} + \abs{w}
\end{align*}
and also take note that:
\begin{align*}
\abs{(w-z) + z} &\leq \abs{w-z} + \abs{z} \\
\abs{w} &\leq \abs{w-z} + \abs{z}
\end{align*}
and, we end up with:
\begin{align*}
\abs{z} - \abs{w} &\leq \abs{z-w} \\
\abs{z} - \abs{w} &\geq -\abs{z-w}
\end{align*}
and combining the preceding results (that is, rearrange last inequality and observe order relations, think about it):
\[ -\abs{z-w} \leq \abs{z} - \abs{w} \leq \abs{z-w}  \]
and this implies:
\[ \abs{\abs{z} - \abs{w}} \leq \abs{z - w} \]
which proves the above theorem.




 



