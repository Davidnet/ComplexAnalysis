\documentclass[notitlepage]{report}
%\usepackage{lmodern}
%\usepackage[T1]{fontenc}
%\usepackage[spanish]{babel}
\usepackage[utf8]{inputenc}
\usepackage{amsmath}
\usepackage{amsfonts}
\usepackage{amssymb}
\usepackage{amsthm}
\author{David Cardozo}
\title{Textbook notes on Complex Analysis}
\newtheorem{define}{Definition}
\newtheorem{thm}{Theorem}

%For Chapter 1
\newcommand{\set}[1]{\left\lbrace #1 \right\rbrace}
\newcommand{\re}[1]{ \mathrm{Re} \left( #1  \right) }
\newcommand{\im}[1]{ \mathrm{Im} \left( #1 \right) }
\newcommand{\cexo }{ \alpha + i \beta}
\newcommand{\cexoc }{ \alpha - i \beta}
\newcommand{\cext }{\gamma + i \delta}
\newcommand{\cextc }{\gamma - i \delta}
\usepackage{mdframed} %frames around examples 
\newcommand{\RR}{\mathbb{R}}
\newcommand{\CC}{\mathbb{C}}
\newcommand{\A}{\alpha}
\newcommand{\B}{\beta}
\newcommand{\abs}[1]{\left| #1 \right|}

\begin{document}
\maketitle
The following are notes based on the book \textit{Complex Analysis} by Ahlfors.
\chapter{Complex Numbers}
\section{The Algebra of Complex Numbers}\label{sec:the-algebra-of-complex-numbers}
We define the imaginary unit as $ i^2 = -1 $, and given two real numbers $ \alpha$ and $ \beta$ we can get the following complex number $ \alpha + i \beta$,  we define the \textbf{real} part of a complex number as: $ \re{ z } = \alpha $, and the \textbf{imaginary} part of a complex number as $ \im{z} = \beta $. If $ \alpha = 0$ the number is said to be \emph{purely imaginary}; if $ \beta = 0$ it is \emph{real}. Observe then, that zero is the only number which is at once real and purely imaginary. Two complex numbers are equal if and only if have the same real part and the same imaginary part.
We define addition and multiplication as follows:
 
\begin{align} \label{Addition}
(\alpha + i \beta) + ( \gamma + i \delta) = (\alpha + \gamma) + i(\beta + \gamma) 
\end{align}

and

\begin{align} \label{multiplication}
(\alpha + i \beta) ( \gamma + i \delta) = ( \alpha \gamma - \beta \delta) + i ( \alpha \delta + \beta \gamma)
\end{align}

Observe that for \eqref{multiplication} we have used the property that $ i ^2 = -1 $

It is less obvious if the operation of division is possible:

\begin{thm}
	\textbf{Existence of Division} 
	$\frac{ \cexo }{ \cext }$ is a complex number, provided that $ \gamma + i \delta \neq 0 $
\end{thm}
\begin{proof}
	Let us denote $ x + i $ the quotient result from the division, for that to happen we must have:
	\begin{align*}
	\cexo = (\cext)(x + iy) 
	\end{align*}
	By \eqref{multiplication} this is equivalent to:
	\begin{align*}
	\cexo = (\gamma x - \delta y ) + i ( \delta x + \gamma y)
	\end{align*}
	And thus, we obtain the equations:
	\begin{align*}
	\alpha = \gamma x - \delta y \\
	\beta =  \delta x + \gamma y
	\end{align*}
	Since this a $ 2 \times 2$ system of equation, it has the unique solution:
	\begin{align*}
	x = \frac{\alpha \gamma  + \beta \delta}{\gamma^2 + \delta^2} \\
	y = \frac{ \beta \gamma - \alpha \delta}{\gamma^2 + \delta^2}
	\end{align*}
	Provided $ \gamma^2 + \delta^2 $ is not zero.
	We have thus the result:
	\begin{align*}
	\frac{ \cexo }{ \cext } =  \frac{\alpha \gamma  + \beta \delta}{\gamma^2 + \delta^2} + i \frac{ \beta \gamma - \alpha \delta}{\gamma^2 + \delta^2} 
	\end{align*}
	Which as for inspection, can be seen as a complex number.
\end{proof}
Once proved that the quotient is a complex number, it is rather cumbersome to compute the quotient in the previous manner, so that we use the following method (Once the existence of the operation is proven):
\begin{align*}
	\frac{\cexo }{\cext  } = \frac{(\cexo )(\cextc )}{(\cext )(\cextc )} = \frac{(\alpha \gamma + \beta) + i(\beta \gamma - \alpha \delta)}{\gamma^2 + \delta^2}
\end{align*} 
Observe then that the \textbf{reciprocal} of a complex number is:
\begin{align*}
\frac{1}{ \alpha + i \beta} = \frac{ \alpha - i \beta}{\alpha^2 + \beta^2}
\end{align*}
We also take the observation that $ i^n $ can only have the four possible values: $1, i, -1, -i$. They correspond to values of $n$ which divided by $4$ leave the remainders $ 0,1,2,3 $.

\begin{mdframed}
	Calculate $i^{-18}$.
	
	\textit{Solution}: Observe that:
	\begin{align*}
	i^{-18} = (i^{18})^{-1}
	\end{align*}
	and then $ x \equiv 18 \mod{4} $ is $ x \equiv 2 \mod{4} $ so that 
	\[ i^{18} = -1 \]
	and finally we have:
	\[ i^{-18} = (-1)^{-1} = -1 \]
	Which is the correct answer.
	
\end{mdframed}
\subsection{Exercises}
\subsubsection{1.}
Find the values of:
\begin{align*}
(1 + 2i)^{3} &= \binom{3}{0}1^3 + \binom{3}{1}1^2 2i + \binom{3}{2}(2i)^2 + \binom{3}{3} (2i)^3 \\
& = 1 + 3 \cdot 2i + 3 \cdot 2^2 i^2 + 1 \cdot 2^3 i^3 \\
& = 1 + 6i - 12 - 8i \\
& = -11 - 2i 
\end{align*}
\rule{\textwidth}{1pt}
\begin{align*}
\frac{5}{-3 + 4i} &= \frac{5}{-3 + 4i} \frac{-3-4i}{-3-4i} \\
 &= \frac{-15 - 20i}{(-3)^{2} -(4i)^{2}} \\
 &= \frac{5 (-3 - 4i)}{5 \cdot 5} \\
 &= \frac{-3-4i}{5} 
\end{align*}
\rule{\textwidth}{1pt}
\begin{align*}
(\frac{2+i}{3-2i})^{2} &= (\frac{2+i}{3-2i} \cdot \frac{3 + 2i}{3 +2i})^{2} \\
 &= (\frac{4 + 7i}{9 + 4 })^2 \\
 &= \frac{56i - 33}{169} 
\end{align*}
\rule{\textwidth}{1pt}
\begin{align*}
(1+i)^n + (1-i)^n &= \sqrt{2}^n e^{\frac{\pi}{4}in} + \sqrt{2}^n e^{-\frac{\pi}{4}in} \\
 & = \sqrt{2}^n(\cos(\frac{\pi}{4}n) + i \sin(\frac{\pi}{4}n) + \cos(-\frac{\pi}{4}n) + i \sin(-\frac{\pi}{4}n)) \\
 & = 2^{\frac{n}{2} + 1} \cos(n \frac{\pi}{4})
\end{align*}
\subsubsection{2.}
If $ z = x + iy $, find the real and imaginary parts of:
Observe that:
\begin{align*}
	z^4 &= (x + iy)^4 \\
	 &= \binom{4}{0}x^4 + \binom{4}{1}x^3iy + \binom{4}{2} x^2(iy)^2 + \binom{4}{3}x(iy)^3 + \binom{4}{4} (iy)^4 \\
	 &= x^4 + 4 x^3iy + (-6) x^2 y^2 + 4 xy^3(-i) + y^4 \\
	 &= x^4 -6x^2y^2 + y^4 + i(4x^3y - 4xy^3)
\end{align*}
And we explicitly see that:
\begin{align*}
\re{z^4} = x^4 -6x^2y^2 + y^4 \quad \im{z^4} = (4x^3y - 4xy^3)
\end{align*}
\rule{\textwidth}{1pt}
We recall the form of $ \frac{1}{z} $ is:
\begin{align*}
\frac{1}{z} = \frac{ x - i y}{x^2 + y^2}
\end{align*}
And we explicitly see that:
\begin{align*}
\re{ \frac{1}{z} } = \frac{ x }{x^2 + y^2} \quad \im{\frac{1}{z}} = \frac{  - y}{x^2 + y^2}
\end{align*}
\rule{\textwidth}{1pt}
Let us take note that
\begin{align*}
\frac{z-1}{z+1} &= \frac{(x -1) + iy}{(x + 1) + iy} \\
&= \frac{((x-1)+iy)((x+1)-iy)}{(x+1)^2 + y^2} \\
&= \frac{(x^2 - 1) + y^2}{(x+1)^2 + y^2} + i\frac{(x+1)y - (x-1)y}{(x+1)^2 + y^2} \\
&= \frac{(x^2 - 1) + y^2}{(x+1)^2 + y^2} + i\frac{2y}{(x+1)^2 + y^2}
\end{align*}
And we explicitly see that:
\begin{align*}
\re{ \frac{z-1}{z+1} } &= \frac{(x^2 - 1) + y^2}{(x+1)^2 + y^2} \\ 
\im{ \frac{z-1}{z+1} } &= \frac{(x+1)y - (x-1)y}{(x+1)^2 + y^2}
\end{align*}
\rule{\textwidth}{1pt}
\begin{align*}
\frac{1}{z^2} &= \frac{1}{(x + iy)^2} \\
 &= \frac{1}{(x^2 - y^2) + 2xyi} \\
 &= \frac{(x^2 - y^2 ) -2xyi}{(x^2 - y^2)^2 - 4x^2y^2}
\end{align*}
So that, we observe:
\begin{align*}
\re{ \frac{1}{z^2} } &= \frac{(x^2 - y^2 )}{(x^2 - y^2)^2 - 4x^2y^2} \\ 
\im{ \frac{1}{z^2} } &= \frac{-2xy}{(x^2 - y^2)^2 - 4x^2y^2}
\end{align*}
\subsubsection{3.}
Show that:
\begin{align*}
(\frac{-1 \pm i \sqrt{3}}{2})^3 = 1
\end{align*}
First, let us observe that:
\begin{align*}
(\frac{-1 \pm i \sqrt{3}}{2})^3 &= \left( \frac{ (2 e^{\pm \frac{\pi}{3}i})}{2} \right)^3 \\
 &= e^{ \pm \pi i} \\
 &= \cos(\pi) + i \sin(\pm \pi) \\
 &= \cos(\pi) \\
 &= 1
\end{align*}
\rule{\textwidth}{1pt}
Consider:
\begin{align*}
\frac{\pm 1 \pm i \sqrt{3}}{2} &= e^{\pm \frac{\pi}{3}} \\
&\text{or} \quad e^{\pm \frac{4 \pi}{3}}
\end{align*}
We consider then two cases:
\begin{align*}
	\left( \frac{\pm 1 \pm i \sqrt{3}}{2} \right)^6  &= \cos( \pm \frac{6 \pi}{3} ) + i \sin( \pm \frac{6 \pi}{3}) \\
	&= \cos(\pm 2 \pi) + i \sin( \pm 2 \pi) \\
	&= \cos( 2 \pi) \\
	&= 1
\end{align*}
Finally,
\begin{align*}
\left( \frac{ \pm 1 \pm i \sqrt{3} }{2} \right) &= \cos( \pm \frac{24 \pi}{3}) + i \sin( \pm \frac{24 \pi}{3}) \\
&= \cos(8 \pi) + i \sin(8 \pi) \\
&= 1
\end{align*}
Which proves the equality.
\section{Square Roots}
The main idea is that we are looking for a number $ x + iy$ such that
\[ 
(x + iy)^{2} = \cexo 
 \]
 Evidently this is same as:
 \begin{align*}
 x^2 - y^2 = \alpha \\
 2xy = \beta
 \end{align*}
Let us take note that
\[ 
(x^{2}+y^{2})^{2} = (x^2 + y^2)^2 + 4x^2y^2
\]
and combining this information along with the previous relationships:
\[ (x^{2}+y^{2})^{2} = \alpha^2 + \beta^2 \]
Hence we should have:
\[ x^2 + y^2 = \sqrt{ \alpha^2 + \beta^2 } \]
and using again the two equalities stated above:
\begin{align*}
x^2 = \frac{1}{2}( \alpha + \sqrt{\alpha^2 + \beta^2}) \\
y^2 = \frac{1}{2}( -\alpha + \sqrt{ \alpha^2 + \beta^2}  )
\end{align*}
Take note, that as much, these two quantities are as much, zero or positive (take extreme values of $\beta$ )
A word of caution the previous equation should be used with caution, since we must be careful to select $x$ and $y$ so that their product has the sign of $\beta$.
In general, but much more cumbersome, a general solution is:
 \begin{align*}
  \sqrt{\cexo }= \pm \left( \sqrt{ \frac{ \alpha + \sqrt{\alpha^2 + \beta^2} }{2}} + i \frac{\beta}{\left| \beta \right|} \sqrt{ \frac{ -\alpha + \sqrt{\alpha^2 + \beta^2}}{2}} \right)
 \end{align*}
 Provided that $ \beta \neq 0 $. As from elementary algebra, we see that for $ \beta = 0$ the values are $ \pm \sqrt{\alpha} $ if $ \alpha \geq 0$, $\pm i \sqrt{-\alpha} $ if $ \alpha < 0$. It is understood that all square rots of positive numbers are taken with the positive sign.
 Observe then that the square root of any complex number exist.
 
 \subsection{Exercises}
\subsubsection{1.}
Compute $\sqrt{i}$, let us observe
\begin{align*}
(x + iy)^2 &= 0 + i \\
x^2 + 2xiy - y^2 = 0 + i
\end{align*}
Which give us the following system:
\begin{align*}
x^2 - y^2 = 0 \\
2xy = 1
\end{align*}
So that 
\begin{align*}
(x^2 + y^2)^2 &= (x^2 - y^2)^2 + 4x^2y^2 \\
&= 1 \\
\end{align*}
so that we have
\begin{align*}
	x^2 &= \frac{1}{2} \\
	y^2 &= \frac{1}{2}
\end{align*}
so that we must set up the possible values of $x$ and $y$ so that $ \beta $ has the right sign.
\begin{align*}
z_1 &= \frac{1}{\sqrt{2}} + i\frac{1}{\sqrt{2}} \\
z_2 &= -\frac{1}{\sqrt{2}} - i\frac{1}{\sqrt{2}}
\end{align*}
\rule{\textwidth}{1pt}
Let us find $x,y$ so that:
\begin{align*}
(x + iy)^2 &= -i \\
x^2 + 2xiy - y^2 = -i
\end{align*}
so that
\begin{align*}
x^2 - y^2 &= -i \\
2xy &= -1 \\
x^2 + y^2 &= 1  
\end{align*}
and we get:
\begin{align*}
x^2 = \frac{1}{2} \\
y^2 = \frac{1}{2}
\end{align*}
and since we need $2xy = -1$, we see that the roots are:
\begin{align*}
z_1 &= \frac{1}{\sqrt{2}} - \frac{1}{\sqrt{2}}i \\
z_2 &= -\frac{1}{\sqrt{2}} + \frac{1}{\sqrt{2}}i
\end{align*}
\rule{\textwidth}{1pt}
\begin{align*}
(x + iy)^2 &= 1 + i \\
x^2 + 2xiy - y^2 &= 1 + i\\
\end{align*}
and we get the system of equalities:
\begin{align*}
x^2 - y^2 = 1 \\ 
2xy = 1
\end{align*}
and after some algebra:
\begin{align*}
x^2 &= \frac{\sqrt{2} + 1}{2} \\
y^2 &= \frac{-1 + \sqrt{2} }{2}
\end{align*}
and having the condition of $x \cdot y > 0 $, we conclude:
\begin{align*}
z_1 &= \sqrt{\frac{\sqrt{2} + 1}{2}} + i \sqrt{\frac{-1 + \sqrt{2}}{2} }\\
z_2 &= -\sqrt{\frac{(\sqrt{2} + 1)}{2}} - i \sqrt{ \frac{-1 + \sqrt{2} }{2} }
\end{align*}
\rule{\textwidth}{1pt}
We want to find $x,y$ so that:
\begin{align*}
(x + iy)^2 = \frac{1 - i \sqrt{3}}{2}
\end{align*}
which give us the system of equalities
\begin{align*}
\begin{cases}
x^2 - y^2 &= \frac{1}{2} \\
2xy &= -\frac{\sqrt{3}}{2} 
\end{cases}
\end{align*}
which ends up with the specification of: 
\begin{align*}
	x^2 &= \frac{3}{4} \\
	y^2 &= \frac{1}{4}
\end{align*}
So that we found:
\begin{align*}
z_1 &= \frac{\sqrt{3}}{2} - i\frac{1}{2} \\
z_2 &= -\frac{\sqrt{3}}{2} + i \frac{1}{2}
\end{align*}
\subsubsection{2.} Find the four values of $ \sqrt[4]{-1} $

We begin then with looking solutions of the form:
\begin{align*}
(x + iy)^4	==ssds
\end{align*} 

 




\section{Roots}
Let us observe the following trick, If $z = r e^{i \theta} $ is a complex number different from zero and $n$ is a positive integer, then there are precisely $n$ different complex numbers $ w_0, w_1, \ldots, w_{n-1}$, that are nth roots of $z$. Let us observe why, let $w$ be a root of $z$ so that:
\begin{align*}
w^n  = z \\
\rho^n e^{in \alpha} = r e^{i \theta} \\
\end{align*}
We conclude then that:
\[ \rho = \sqrt[n]{r} \]
 is the real, positive nth root of $z$, but for the angle we must be careful, we can only say that
 \[ n\alpha = \theta + 2\pi k  \] 
 $k = \pm 1, \pm 2, \pm 3 .... $ so that we end up with:
 \[ n = \frac{\theta}{\alpha} + k\frac{2\pi}{\alpha} \]
 it may seem that there are infinite solutions, but for the sake of easiness we decide to take 
 \[ k = 0, \ldots, n-1 \]
 All the nth roots of any complex number lie on a circle centered at the origin and having radius equal to the real, positive nth root of r. One of them has argument $ \alpha = \frac{\theta}{n} $. The others are uniformly spaced around the circle, each being separeted from its neighbors by an angle equal to $ \frac{2\pi}{n} $
\section*{A bit of Analysis}
Now if we move a little bit to analysis, we have the following theorem.

Preceding inequalities to remember:

\[ \abs{\operatorname{Re}(z)} \leq \abs{z} \quad \abs{\operatorname{Im}(z)} \leq \abs{z} \]

and an equality to remember:

\[ \abs{z}^2 = z \cdot \bar{z} \]

\textbf{Propositon} The complex sequence $ \set{z_n} $, where $ z_n = x_n + iy_n $, converges to the limit $ \zeta = \xi + i\eta $, if and only if the real sequences $ \set{x_n}  $ and $ \set{y_n} $ converges to the limits $ \xi $ and $ \eta $ respectively.

\textit{Proof} Suppose that $ z_n \rightarrow \zeta  $, that is, for any $ \epsilon > 0 $, there exist $ N \in \mathbb{N} $, such that $ \abs{z_n - \zeta}<\epsilon $ holds for $ n > N$. It follows then that:
\begin{align*}
\abs{x_n - \xi} = \abs{\operatorname{Re}(z_n - \zeta)} \leq \abs{z_n - \zeta} < \epsilon \\
\abs{y_n - \eta} = \abs{\operatorname{Im}(z_n - \zeta)} \leq \abs{z_n - \zeta} < \epsilon
\end{align*}
and these holds for $n > N $, and we conclude that $ x_n \rightarrow \xi $ and $ y_n \rightarrow \eta $.

Now conversely, suppose that $ x_n \rightarrow \xi $, and $ y_n \rightarrow \eta $, or in other words, for every $ \epsilon > 0 $, there exists an $ N \in \mathbb{N} $, such that $ \abs{x_n - \xi} < \epsilon $ and $ \abs{y_n - \eta} < \epsilon $ for $ n > N $. It follows then:
\[ \abs{z_n - \zeta} = \abs{(x_n - \xi) + i(y_n - \eta)} \overset{\Delta-\textrm{inequality}}{\leq} \abs{x_n - \xi} + \abs{y_n - \eta} < \epsilon \]
holds for $ n > N $ and hence $ z_n \rightarrow \zeta $ as $ n \rightarrow \infty $. \qed

A sequence $ \set{z_n} $ is said to be \textbf{Cauchy sequence} if:
\[ \abs{z_n - z_m} \rightarrow 0 \quad \textrm{ as } n,m \rightarrow \infty  \]

more formally:
\begin{define}
	A sequence $ \set{z_n} $ is said to be a \textbf{Cauchy sequence }, if given $ \epsilon > 0 $ there exists an integer $ N > 0 $ so that $ \abs{z_n - z_m} < \epsilon $ whenever $ n,m > N $
\end{define}

\textbf{Proposition} $z_n$ is Cauchy if and only if the real and imaginary parts are Cauchy.

\textit{Proof} Suppose $ z_n $ is Cauchy, so that there exist $N \in \mathbb{N} $ for which:
\[ \abs{z_n - z_m} < \epsilon \]
holds for $ n,m > N $, so observe that:
\begin{align*}
\abs{\operatorname{Re}(z_n - z_m)} \leq \abs{z_n - z_m} &< \epsilon \\
\abs{\operatorname{Im}(z_n - z_m)} \leq \abs{z_n - z_m} &< \epsilon 
\end{align*}

Conversely, suppose that real and imaginary parts are Cauchy, that is, $ \abs{\operatorname{Re}(z_n - z_m)} < \frac{\epsilon}{2} $ and $ \abs{\operatorname{Im}(z_n - z_m)}< \frac{\epsilon}{2} $ holds for some $ N \in \mathbb{N} $ and $ n,m > N  $, so let us observe:
\begin{align*}
\abs{z_n - z_m } = \abs{\operatorname{Re}(z_n - z_m) + i (\operatorname{Im}(z_n - z_m))} \\
\abs{\operatorname{Re}(z_n - z_m) + i (\operatorname{Im}(z_n - z_m))}  \leq \abs{\operatorname{Re}(z_n - z_m)} + \abs{\operatorname{Im}(z_n - z_m)}
\end{align*}
so that:
\begin{align*}
\abs{z_n - z_m} \leq \frac{\epsilon}{2} + \frac{\epsilon}{2} = \epsilon
\end{align*}
so we conclude that $ z_n - z_m $ is Cauchy. \qed

Since the sequence $ \set{z_n} $ is Cauchy if and only if the sequences of real and imaginary parts of $ z_n $ are, we conclude that every Cauchy sequence in $ \CC $ has a limit in $ \CC $, summarizing:

\begin{thm}
	$ \CC $ is complete
\end{thm}

\textbf{ \LARGE Sets in the complex plane}

\begin{define}
	We define the \textbf{open disc} $ D_r(z_0) $ \textbf{of radius} $r$ \textbf{centered at} $ z_0 $ to be the set of all complex numbers that are at absolute value strictly less than $r$ from $ z_0 $. In other words,
	\[ D_r(z_0) = \set{z \in \CC : \abs{z - z_0}< r} \]
\end{define}

Analogously, the \textbf{closed disc}:
\[ \bar{D}_r(z_0) = \set{z \in \CC : \abs{z - z_0} \leq r} \]

The \textbf{unit disc} (open disc centered at the origin with radius $1$) has an important notation:
\[ \mathbb{D} = \set{z \in \CC : \abs{z}< 1} \]

\begin{define}
	Given $ \Omega \subset \CC $, a point $ z_0 $ is an \textbf{interior point} of $ \Omega $ if there exists $ r>0  $ such that:
	\[ D_r(z_0) \subset \Omega \]
\end{define}

The \textbf{interior} of $ \Omega $ consists of all its interior points. Finally, a set $ \Omega $ is \textbf{open} if every point in that set is an interior point of $ \Omega $.

A set $ \Omega $ is \textbf{closed} if its complement $ \CC - \Omega $ is open.

\begin{define}
	A point $ z \in \CC $ is said to be a \textbf{limit point} of the set $ \Omega $ if there exists a sequence of points $ z_n \in \Omega $ such that $ z_n \neq z $ and $ \lim\limits_{n \rightarrow \infty} z_n = z $
\end{define}

\textbf{Proposition} A set is closed if and only if it contains all its limit points.

\textit{Proof} 
Suppose that $\Omega$ is closed, so that $ \Omega^c $ is open, since $ \Omega^c $ is open, every point of $ \Omega^c $ is an interior point




\end{document}