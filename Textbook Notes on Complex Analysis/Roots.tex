\section{Roots}
Let us observe the following trick, If $z = r e^{i \theta} $ is a complex number different from zero and $n$ is a positive integer, then there are precisely $n$ different complex numbers $ w_0, w_1, \ldots, w_{n-1}$, that are nth roots of $z$. Let us observe why, let $w$ be a root of $z$ so that:
\begin{align*}
w^n  = z \\
\rho^n e^{in \alpha} = r e^{i \theta} \\
\end{align*}
We conclude then that:
\[ \rho = \sqrt[n]{r} \]
 is the real, positive nth root of $z$, but for the angle we must be careful, we can only say that
 \[ n\alpha = \theta + 2\pi k  \] 
 $k = \pm 1, \pm 2, \pm 3 .... $ so that we end up with:
 \[ n = \frac{\theta}{\alpha} + k\frac{2\pi}{\alpha} \]
 it may seem that there are infinite solutions, but for the sake of easiness we decide to take 
 \[ k = 0, \ldots, n-1 \]
 All the nth roots of any complex number lie on a circle centered at the origin and having radius equal to the real, positive nth root of r. One of them has argument $ \alpha = \frac{\theta}{n} $. The others are uniformly spaced around the circle, each being separeted from its neighbors by an angle equal to $ \frac{2\pi}{n} $